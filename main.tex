\documentclass[a4paper]{article}

%% Language and font encodings
\usepackage[english]{babel}
\usepackage[utf8x]{inputenc}
\usepackage[T1]{fontenc}
\usepackage{array}
\usepackage{amsmath}
\usepackage{array}
\newcolumntype{C}{>$c<$}

%% Sets page size and margins.
\usepackage{amsfonts,amssymb}
\usepackage[margin=1in]{geometry}


\title{Math 312\\Theorem 71}
\author{Chachi Lor}
\date{}

\begin{document}
\maketitle
\underline{Proof.}

Let there be a decreasing sequence of closed bounded intervals,

\begin{center}

$[a_1,b_1] \supset [a_2,b_2] \supset \dots \supset [a_n,b_n] = I_1 \supset I_2 \supset \dots \supset I_n$ such that\\

$a_1 \leq a_2 \leq \dots \leq a_n < b_n \leq \dots \leq b_2 \leq b_1.$

\end{center}
This shows that we have an increasing sequence $\{ a_n \}$ and a decreasing sequence $\{ b_n \}$ and by 
theorem 63 and 64, $\{ a_n \}$ converges to $a$ and $\{ b_n \}$ converges to $b$, since $a_n < b_n$ then b is a upper bound of $\{ a_n \}$ therefore by Axiom II the supremum for $a_n$ would be $a$, $a \leq b$ therefore $a_n \leq b \leq b_n$. Since $b \in I_n$ then we know that $b \in \displaystyle\bigcap_{n=1}^\infty I_n$ because it is the only point that is contained in all the intervals. $\Box$

\end{document}
